\documentclass[a4paper, 11pt, finnish]{article}
\usepackage{ucs}
\usepackage[utf8x]{inputenc}
\usepackage[T1]{fontenc}
\usepackage[finnish]{babel}
\usepackage{graphicx}
\usepackage{hyperref}

\setlength{\parindent}{0pt}
\setlength{\parskip}{1ex plus 0.5ex minus 0.2ex}

\author{Topi Talvitie}
\title{frivol: Käyttöohje}

\begin{document}
\maketitle

\section*{frivol}
frivol-kirjaston käyttö on helppoa, sillä se koostuu pelkästään headeritiedostoista, eli mitään DLL:iä ei tarvitse liittää. Kirjasto löytyy frivol/-hakemistosta, ja vaatii että kyseinen hakemisto on kääntäjän headerinetsintäpolussa (esimerkiksi GCC:ssä ''-I<hakemisto joka sisältää frivol/-hakemiston>''). Kirjasto on C++11-kirjasto, joten käännös pitää suorittaa C++11-tilassa (esimerkiksi GCC:ssä ''-std=c++0x'').

Funktio, jolla saa laskettua annetun pistejoukon Voronoi-diagrammin löytyy tiedostosta frivol/frivol.hpp. Kirjasto ei vielä käsittele kaikkia mahdollisia syötteitä oikein (ks. toteutusdokumentti -> Puutteet). Tämän takia ei ole suositeltavaa käyttää kirjastoa virheille herkissä tarkoituksissa.

\section*{test ja perftest}
Testikokoelma ja suorituskykytestiohjelma löytyvät hakemistoista test/ ja perftest/. Nämä käännetään CMakella, hakemistojen README.md:ssä tarkemmat ohjeet.

\section*{frivoldraw}
frivoldraw-visualisaatio löytyy hakemistosta frivoldraw/, ja sen kääntäminen vaatii Emscripten-asennuksen toimiakseen. Tarkemmat tiedot hakemiston README.md:ssä.
\end{document}

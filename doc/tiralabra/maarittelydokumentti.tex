\documentclass[a4paper, 11pt, finnish]{article}
\usepackage{ucs}
\usepackage[utf8x]{inputenc}
\usepackage[T1]{fontenc}
\usepackage[finnish]{babel}

\setlength{\parindent}{0pt}
\setlength{\parskip}{1ex plus 0.5ex minus 0.2ex}

\author{Topi Talvitie}
\title{frivol: Määrittelydokumentti}

\begin{document}
\maketitle

Pistejoukon Voronoi-diagrammilla tarkoitetaan tason jakoa maksimaalisiin alueisiin siten, että kaikilla saman alueen pisteillä on sama lähin piste syötepistejoukosta. Voronoi-diagrammin alueet ovat aina monikulmioita, mahdollisesti rajoittamattomia.

Projektissa toteutetaan kirjasto annetun pistejoukon Voronoi-diagrammien laskentaan. Kirjaston nimi on \emph{frivol} eli Friendly Voronoi Diagram Library. Tämän lisäksi toteutetaan ohjelma \emph{frivoltool} joka kirjastoa käyttäen laskee annetun pistejoukon Voronoi-diagrammin ja toinen ohjelma \emph{frivoldraw}, joka osaa piirtää kuvan Voronoi-diagrammista.

\section*{Syöte ja tuloste}
Kirjaston rajapinta on vain yksi funktio, joka ottaa syötteenä taulukon tason pisteitä ja palauttaa taulukon, jossa on vastaavat Voronoi-diagrammin alueet. Alue sisältää listan alueen rajoitetuista kärkipisteistä, ja mikäli alue on rajoittamaton, myös molempien rajoittamattomien sivujen suuntavektorit. Alueiden sivuihin tallennetaan lisäksi naapurimonikulmioiden indeksit.

\emph{frivoltool} on yksinkertainen ohjelma joka käärii kirjaston funktion komentorivikäyttöön sopivaksi, eli syöte saadaan ja tuloste annetaan merkkijonomuodossa.

\emph{frivoldraw} ottaa syötteen samalla tavalla kuin \emph{frivoltool}, mutta tulosteen kirjoittamisen merkkijonomuodossa sijaan se piirtää siitä kuvan, parametrien mukaan joko suoraan avattavaan ikkunaan tai kuvatiedostoon. Kuvassa näkyy pistejoukko ja Voronoi-diagrammin alueiden reunat.

\section*{Algoritmi}


\end{document}

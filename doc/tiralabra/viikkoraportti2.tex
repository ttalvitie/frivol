\documentclass[a4paper, 11pt, finnish]{article}
\usepackage{ucs}
\usepackage[utf8x]{inputenc}
\usepackage[T1]{fontenc}
\usepackage[finnish]{babel}
\usepackage{graphicx}
\usepackage{hyperref}

\setlength{\parindent}{0pt}
\setlength{\parskip}{1ex plus 0.5ex minus 0.2ex}

\author{Topi Talvitie}
\title{frivol: Viikkoraportti 2}

\begin{document}
\maketitle

Toisen viikon alku kului koodia uudelleenjärjestellessä. Kooditiedostot jaettiin hakemistorakenteeseen ja sitä vastaavaan namespacerakenteeseen. Lisäksi koska fortune::Algorithm-luokka oli aika monimutkainen, sen ''beach line''-osat siirrettiin omaan luokkaansa. Kävin myös ohjauksessa puhumassa viime palautuksesta saadusta palautteesta, jonka osasta olen yhä eri mieltä - mielestäni vaikka fortune::Algorithmissa on paljon kommentointia funktioiden toteutuksessa, on se välttämätöntä algoritmin luettavuuden kannalta. Jotkin asiat eivät vain selviä niin nopeasti koodista kuin kuvailusta ja tällöin on yhtenäisen tyylin kannalta hyvä ainakin samassa funktiossa sanallisesti kuvailla kaikki vaiheet.

Uuden hiukan erilaisen BeachLine-toteutuksen kanssa ilmeni omituisia ongelmia, jotka paikantuivat GeometryTraitsien float-toteutusten getBreakpointX-funktioon, joka ei hoitanut kaikkia erikoistapauksia oikein. Uuden, numeerisesti stabiilimman ja selkeämmän version kanssa ei ole ilmennyt ongelmia, mutta tässä vaiheessa alkaa olla selvää, että kuten useimmissa laskennallisen geometrian algoritmeissa, myös Fortunen algoritmissa joutuu painimaan numeeristen ongelmien kanssa. Esimerkiksi ensimmäistä kertaa minulle aiheutui käytännön ongelmia siitä, että lukiossa opetettu toisen asteen yhtälön ratkaisukaava on nollaa lähellä oleville ratkaisuille numeerisesti epästabiili. Uusien osien toimivuuden varmistamiseksi toteutin lisää testejä etenkin GeometryTraitseille sekä BeachLinelle.

Kovin paljon uusia ominaisuuksia koodiin ei siis tällä viikolla tullut, mutta kokonaisuutena katsottuna projekti eteni hyvää tahtia. Seuraavaksi pitäisi toteuttaa Fortunen algoritmin tulosteen (diagrammin alueet, särmät ja solmut) muodostaminen, ja sitten lähteä toteuttamaan tietorakenteita Dummy-tietorakenteiden tilalle. Lisäksi joitain koodinhoidollisia toimia tarvitaan - esimerkiksi GeometryTraitseissa on vielä joitain osia joiden koodi on vain nopeasti tuotettu Maplesta saaduista kaavoista jotta voitaisiin toteuttaa ydinalue ensin loppuun.

\end{document}

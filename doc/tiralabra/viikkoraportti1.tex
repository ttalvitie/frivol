\documentclass[a4paper, 11pt, finnish]{article}
\usepackage{ucs}
\usepackage[utf8x]{inputenc}
\usepackage[T1]{fontenc}
\usepackage[finnish]{babel}
\usepackage{graphicx}
\usepackage{hyperref}

\setlength{\parindent}{0pt}
\setlength{\parskip}{1ex plus 0.5ex minus 0.2ex}

\author{Topi Talvitie}
\title{frivol: Viikkoraportti 1}

\begin{document}
\maketitle

Ensimmäisellä viikolla toteutin tarvittavia perustyökaluja (Stack, Array) sekä tietorakenteiden (prioriteetijono ja hakupuu) rajapinnat ja ne tarkistavat konseptit. Toteutin myös tietorakennekonseptit mahdollisimman yksinkertaisilla tavoilla (DummyPriorityQueue ja DummySearchTree). Tein näille myös tarvittavat testit.

Tämän jälkeen aloin toteuttamaan Fortunen algoritmia, ja tein siihen tarvittavia lisätyökaluja, kuten GeometryTraits. Viikon 1 palautukseen sain siis valmiiksi Fortunen algoritmin, joka osaa laskea Voronoi-solmujen lukumäärän. Algoritmin toimivuus osoitetaan siis tässä vaiheessa sillä, että se osaa laskea Voronoi-solmujen lukumäärän oikein testipaketissa fortune\_algorithm.

Fortunen algoritmin lisäksi tällä viikolla uusia asioita minulle ovat olleet Boost-konseptit. Vaikka todennäköisesti ilman konseptitarkistuksia pärjäisi hyvin, ovat konseptit luonnollinen paikka kirjoittaa rajapinnan dokumentaatio ja ne antavat parempia virheilmoituksia jos toteutuksessa on vääränlainen rajapinta.

Projekti on siis edistynyt aika hyvin, enää seuraavat asiat täytyy toteuttaa:
\begin{enumerate}
\item Fortunen algoritmiin diagrammin koostaminen
\item PriorityQueueConceptin ja SearchTreeConceptin toteutukset Dummy-toteutusten tilalle käyttäen binäärikekoa ja tasapainotettua binääripuuta.
\item Kirjastoa käyttävät esimerkkiohjelmat \emph{frivoltool} ja \emph{frivoldraw}.
\end{enumerate}
Aion toteuttaa asiat listan järjestyksessä.



\end{document}

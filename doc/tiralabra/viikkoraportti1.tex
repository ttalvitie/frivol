\documentclass[a4paper, 11pt, finnish]{article}
\usepackage{ucs}
\usepackage[utf8x]{inputenc}
\usepackage[T1]{fontenc}
\usepackage[finnish]{babel}
\usepackage{graphicx}
\usepackage{hyperref}

\setlength{\parindent}{0pt}
\setlength{\parskip}{1ex plus 0.5ex minus 0.2ex}

\author{Topi Talvitie}
\title{frivol: Viikkoraportti 1}

\begin{document}
\maketitle

Ensimmäisellä viikolla toteutin tarvittavia perustyökaluja (Stack, Array) sekä tietorakenteiden (prioriteetijono ja hakupuu) rajapinnat ja ne tarkistavat konseptit. Tein näille myös tarvittavat testit. Sitten aloin toteuttamaan Fortunen algoritmia, ja tein siihen tarvittavia lisätyökaluja, kuten GeometryTraits. Viikon 1 palautukseen sain siis valmiiksi Fortunen algoritmin, joka osaa laskea Voronoi-solmujen lukumäärän. Algoritmin toimivuus osoitetaan siis tässä vaiheessa sillä, että se osaa laskea Voronoi-solmujen lukumäärän oikein testipaketissa fortune\_algorithm.

\end{document}
